\PassOptionsToPackage{unicode=true}{hyperref} % options for packages loaded elsewhere
\PassOptionsToPackage{hyphens}{url}
%
\documentclass[
]{article}
\usepackage{lmodern}
\usepackage{amssymb,amsmath}
\usepackage{ifxetex,ifluatex}
\ifnum 0\ifxetex 1\fi\ifluatex 1\fi=0 % if pdftex
  \usepackage[T1]{fontenc}
  \usepackage[utf8]{inputenc}
  \usepackage{textcomp} % provides euro and other symbols
\else % if luatex or xelatex
  \usepackage{unicode-math}
  \defaultfontfeatures{Scale=MatchLowercase}
  \defaultfontfeatures[\rmfamily]{Ligatures=TeX,Scale=1}
\fi
% use upquote if available, for straight quotes in verbatim environments
\IfFileExists{upquote.sty}{\usepackage{upquote}}{}
\IfFileExists{microtype.sty}{% use microtype if available
  \usepackage[]{microtype}
  \UseMicrotypeSet[protrusion]{basicmath} % disable protrusion for tt fonts
}{}
\makeatletter
\@ifundefined{KOMAClassName}{% if non-KOMA class
  \IfFileExists{parskip.sty}{%
    \usepackage{parskip}
  }{% else
    \setlength{\parindent}{0pt}
    \setlength{\parskip}{6pt plus 2pt minus 1pt}}
}{% if KOMA class
  \KOMAoptions{parskip=half}}
\makeatother
\usepackage{xcolor}
\IfFileExists{xurl.sty}{\usepackage{xurl}}{} % add URL line breaks if available
\IfFileExists{bookmark.sty}{\usepackage{bookmark}}{\usepackage{hyperref}}
\hypersetup{
  pdftitle={Population Ecology HW1},
  pdfauthor={Tess Hooper},
  pdfborder={0 0 0},
  breaklinks=true}
\urlstyle{same}  % don't use monospace font for urls
\usepackage[margin=1in]{geometry}
\usepackage{graphicx,grffile}
\makeatletter
\def\maxwidth{\ifdim\Gin@nat@width>\linewidth\linewidth\else\Gin@nat@width\fi}
\def\maxheight{\ifdim\Gin@nat@height>\textheight\textheight\else\Gin@nat@height\fi}
\makeatother
% Scale images if necessary, so that they will not overflow the page
% margins by default, and it is still possible to overwrite the defaults
% using explicit options in \includegraphics[width, height, ...]{}
\setkeys{Gin}{width=\maxwidth,height=\maxheight,keepaspectratio}
\setlength{\emergencystretch}{3em}  % prevent overfull lines
\providecommand{\tightlist}{%
  \setlength{\itemsep}{0pt}\setlength{\parskip}{0pt}}
\setcounter{secnumdepth}{-2}
% Redefines (sub)paragraphs to behave more like sections
\ifx\paragraph\undefined\else
  \let\oldparagraph\paragraph
  \renewcommand{\paragraph}[1]{\oldparagraph{#1}\mbox{}}
\fi
\ifx\subparagraph\undefined\else
  \let\oldsubparagraph\subparagraph
  \renewcommand{\subparagraph}[1]{\oldsubparagraph{#1}\mbox{}}
\fi

% set default figure placement to htbp
\makeatletter
\def\fps@figure{htbp}
\makeatother


\title{Population Ecology HW1}
\author{Tess Hooper}
\date{1/15/2020}

\begin{document}
\maketitle

\hypertarget{question-1.-why-might-a-plant-or-animal-be-patchily-distributed}{%
\section{Question 1. Why might a plant or animal be patchily
distributed?}\label{question-1.-why-might-a-plant-or-animal-be-patchily-distributed}}

\begin{enumerate}
\def\labelenumi{\arabic{enumi}.}
\item
  Availability of resources (e.g.~water, food) might affect where plants
  or animals are distributed. In this case I would expect the location
  of high-density patches to be inconsistent from year to year, as the
  resources may change. For instance, if certain foraging plants are
  abundant in one area, then that is where we might find ungulates in a
  given year. However, if they deplete that resource then that group of
  ungulates may need to move locations.
\item
\item
\end{enumerate}

\begin{verbatim}
## Warning: package 'tidyverse' was built under R version 3.6.2
\end{verbatim}

\begin{verbatim}
## Warning: package 'ggplot2' was built under R version 3.6.2
\end{verbatim}

\begin{verbatim}
## Warning: package 'tibble' was built under R version 3.6.2
\end{verbatim}

\begin{verbatim}
## Warning: package 'tidyr' was built under R version 3.6.2
\end{verbatim}

\begin{verbatim}
## Warning: package 'readr' was built under R version 3.6.2
\end{verbatim}

\begin{verbatim}
## Warning: package 'purrr' was built under R version 3.6.2
\end{verbatim}

\begin{verbatim}
## Warning: package 'dplyr' was built under R version 3.6.2
\end{verbatim}

\begin{verbatim}
## Warning: package 'stringr' was built under R version 3.6.2
\end{verbatim}

\begin{verbatim}
## Warning: package 'forcats' was built under R version 3.6.2
\end{verbatim}

\hypertarget{question-2-eureaka-dune-grass}{%
\section{Question 2: Eureaka Dune
Grass}\label{question-2-eureaka-dune-grass}}

\begin{verbatim}
## 
##  Welch Two Sample t-test
## 
## data:  swallenia2$count_2009 and swallenia2$count_2010
## t = -0.81791, df = 17.37, p-value = 0.4245
## alternative hypothesis: true difference in means is not equal to 0
## 95 percent confidence interval:
##  -141.06385   62.15476
## sample estimates:
## mean of x mean of y 
##  61.90909 101.36364
\end{verbatim}

C. Which analysis is more appropriate?

D. Paragraph to the park superintendant describing my findings

\hypertarget{question-3-yellowstone-grizzly-bears}{%
\section{Question 3: Yellowstone Grizzly
Bears}\label{question-3-yellowstone-grizzly-bears}}

\hypertarget{question-2.-eureka-dune-grass-populations}{%
\section{Question 2. Eureka Dune Grass
Populations}\label{question-2.-eureka-dune-grass-populations}}

\hypertarget{a.}{%
\subsection{a.}\label{a.}}

\end{document}
